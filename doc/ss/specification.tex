\section{Specifications}
\label{sec:specifications}
The structure behind the product is described in  section~\ref{sec:overall_architecture}, where the required entities are modelled along with there relationships. The different layers of the application architecture is also described along with a navigation map of the different user interfaces. Each system feature will, in section~\ref{sec:system_features}, be examined and described. All the required user interfaces are described along with their respective system feature. In section~\ref{sec:directory_organization} is the physical file structure of the project described. In the last section, ~\ref{sec:class_description}, are all classes listed along with a description, what classes are imported and their location in the project.


\subsection{Overall Architecture}
\label{sec:overall_architecture}
The overall architecture of the project is built as an \textit{Android} application; the user interfaces are built up by so called \textit{Activities} while any background tasks are run as \textit{Services}. All background functionality is built up by standard \textit{Java} classes though some of them inherit from \textit{Android} classes.
\begin{figure}[hbt]
\centering
\includegraphics[width=0.8\textwidth]{./images/ApplicationLayers.png}
\caption{Layers of the Application Architecture}
\label{fig:app_layers}
\end{figure}
To allow the user to input information such as category names or feed links \textit{Dialogs} are used, based on the \textit{Android} class \textit{DialogFragment}. The complete architecture of the application is shown in figure~\ref{fig:app_layers} divided into different layers. The two top layers are the actual application, \textit{User Interfaces} and \textit{Backend}, and utilizes the middle layer \textit{Utilities}. These give fully access and control over the lower layer containing the \textit{Entities}. In the current architecture all the \textit{Utilities} are connected to all the types of \textit{Entities}. 
\begin{figure}[hbt]
\centering
\includegraphics[width=0.8\textwidth]{./images/EntityClassDiagram.png}
\caption{Entity Class Diagram of the Product}
\label{fig:entity}
\end{figure}
The \textit{Entities} of the application are \textit{Category}, \textit{Feed} and \textit{Post} and are shown in figure~\ref{fig:entity}. Here all sub-entities are included and their relationship to each other.

\newpage
\subsection{System Features}
\label{sec:system_features}
In this section are all the included system features examined and described with the help of their respective class diagram.


\subsubsection{Share Post}
\begin{figure}[hbt]
\centering
\includegraphics[width=0.45\textwidth]
{./images/SharePost.png}
\caption{Class Diagram of system feature Share Post}
\label{fig:share}
\end{figure}
The ability to share a post requires two things, a post to share and a target to share it to. The relationship required to perform this feature is visualized in figure~\ref{fig:share}. Here the user has the ability through the user interface to share a post to a social medium of its choice. The social medium is accessed through an interface to its API. \\
\begin{tabular}{l l}
\begin{minipage}{0.5\textwidth}
\begin{enumerate}
  \item Access the post view, shown in figure~\ref{fig:share_screen}.
  \item Display all actions by pressing the \textit{Menu} button on the device.
  \item Press \textit{Share Post}.
  \item In the displayed dialog enter the required credentials to be able to share the post the chosen medium.
\end{enumerate}
\end{minipage}
&
\begin{minipage}{0.5\textwidth}
  \centering
  \includegraphics[width=0.8\textwidth]{./images/PostOptions.png}
  \captionof{figure}{Screenshot of the post view}
  \label{fig:share_screen}
\end{minipage}
\end{tabular}


\newpage
\subsubsection{Count Un-read Post(s)}
\begin{figure}[hbt]
\centering
\includegraphics[width=0.6\textwidth]
{./images/CountUnreadPosts.png}
\caption{Class Diagram of system feature Count Un-read Post(s)}
\label{fig:count}
\end{figure}
To count un-read post(s) access to not only the posts but also the feeds and the feed list are required so that posts of all feeds are accounted for. The relationship required to perform this feature is visualized in figure~\ref{fig:count}. Here the user accesses the number of un-read post(s) through the user interface which in turn is connected to all the required classes to determine the total un-read post(s) count. \\
\begin{tabular}{l l}
\begin{minipage}{0.5\textwidth}
\begin{enumerate}
  \item Access the feed view, shown in figure~\ref{fig:count_screen}.
  \item Unread posts are shown as bold in the list, while read posts are not bold.
\end{enumerate}
\textit{The total amount of unread posts can be accessed internally through a database query, and implemented to be shown where it fits.}
\end{minipage}
&
\begin{minipage}{0.5\textwidth}
  \centering
  \includegraphics[width=0.8\textwidth]{./images/ViewReadAndUnreadPosts.png}
  \captionof{figure}{Screenshot of the feed view}
  \label{fig:count_screen}
\end{minipage}
\end{tabular}


\newpage
\subsubsection{Read Post(s)}
\begin{figure}[hbt]
\centering
\includegraphics[width=0.45\textwidth]
{./images/ReadPosts.png}
\caption{Class Diagram of system feature Read Post(s)}
\label{fig:read}
\end{figure}
For users to access and read post(s) an user interface is required that is connected to the actual post(s). The relationship required to perform this feature is visualized in figure~\ref{fig:read}. The user interface will determines which post(s) based on previous state, e.g. which post the user has selected to read coming from a previous view. \\
\begin{tabular}{l l}
\begin{minipage}{0.5\textwidth}
\begin{enumerate}
  \item Access the post view, shown in figure~\ref{fig:post_screen}.
  \item Be notified by the interface that the post has been marked as read.
  \item Read title, publication date and source feed in the header.
  \item Read the description, that may include pictures, below the header.
  \item To access the actually post through a web browser click the title in the header.
\end{enumerate}
\end{minipage}
&
\begin{minipage}{0.5\textwidth}
  \centering
  \includegraphics[width=0.8\textwidth]{./images/MarkPostRead.png}
  \captionof{figure}{Screenshot of the post view}
  \label{fig:post_screen}
\end{minipage}
\end{tabular}


\newpage
\subsubsection{View Feed(s)}
\begin{figure}[hbt]
\centering
\includegraphics[width=0.6\textwidth]
{./images/ViewFeeds.png}
\caption{Class Diagram of system feature View Feed(s)}
\label{fig:view}
\end{figure}
Users access and view feed(s) through an user interface that is connected to the feed(s). The relationship required to perform this feature is visualized in figure~\ref{fig:view}. The user interface determines which feed(s) based on previous state, e.g. which category of feeds the user has selected to view coming from a previous view. \\
\begin{tabular}{l l}
\begin{minipage}{0.5\textwidth}
\begin{enumerate}
  \item Access the feed view, shown in figure~\ref{fig:feed_screen}.
  \item The 30 latest posts are shown in a scrollable list.
  \item Press a post to access the post view for that specific post.
\end{enumerate}
\end{minipage}
&
\begin{minipage}{0.5\textwidth}
  \centering
  \includegraphics[width=0.8\textwidth]{./images/ViewPosts.png}
  \captionof{figure}{Screenshot of the feed view}
  \label{fig:feed_screen}
\end{minipage}
\end{tabular}


\newpage
\subsubsection{View Statistics}
\begin{figure}[hbt]
\centering
\includegraphics[width=0.6\textwidth]
{./images/CountUnreadPosts.png}
\caption{Class Diagram of system feature View Statistics}
\label{fig:statistics}
\end{figure}
To view statistics access to not only the posts but also the feeds and the feed list are required so that all statistics on all levels are collected. The relationship required to perform this feature is visualized in figure~\ref{fig:count}. Here the user accesses the statistics through the user interface which in turn is connected to all the required classes to collect all interesting data. \\
\begin{tabular}{l l}
\begin{minipage}{0.5\textwidth}
\begin{enumerate}
  \item Access any view of feed level or higher, shown in figure~\ref{fig:feed_opt_screen}.
  \item Display all actions by pressing the \textit{Menu} button on the device.
  \item Press \textit{Statistics}.
  \item In the displayed dialog statistics are shown for the current level.
\end{enumerate}
\end{minipage}
&
\begin{minipage}{0.5\textwidth}
  \centering
  \includegraphics[width=0.8\textwidth]{./images/FeedOptions.png}
  \captionof{figure}{Screenshot of the feed view}
  \label{fig:feed_opt_screen}
\end{minipage}
\end{tabular}


\newpage
\subsubsection{Import/Export Feed(s)}
\begin{figure}[hbt]
\centering
\includegraphics[width=0.5\textwidth]
{./images/ImportExportFeeds.png}
\caption{Class Diagram of system feature Import/Export Feed(s)}
\label{fig:import}
\end{figure}
By an interface to a storage area, external memory in the case of Android, the user has the possible to either import a set of feed(s) or export. The relationship required to perform this feature is visualized in figure~\ref{fig:import}. Both actions are triggered through the user interface. For importing the user chooses a file on external memory, with a standardized file format for feed lists, and the product performs a mass add of the feeds in the file. An export action from the user will create a feed list file, with the same standardized file format as mentioned earlier. \\
\begin{tabular}{l l}
\begin{minipage}{0.5\textwidth}
\begin{enumerate}
  \item Access the main view, shown in figure~\ref{fig:import_screen}.
  \item Display all actions by pressing the \textit{Menu} button on the device.
  \item Press \textit{Import/Export Feeds}.
  \item In the displayed dialog chose if you want to import or export feeds.
  \item In the following dialog chose either which file to import or what file name to export to.
\end{enumerate}
\end{minipage}
&
\begin{minipage}{0.5\textwidth}
  \centering
  \includegraphics[width=0.8\textwidth]{./images/ImportExport.png}
  \captionof{figure}{Screenshot of the main view}
  \label{fig:import_screen}
\end{minipage}
\end{tabular}


\newpage
\subsubsection{Find Similar Feed(s)}
\begin{figure}[hbt]
\centering
\includegraphics[width=0.5\textwidth]
{./images/FindSimilarFeeds.png}
\caption{Class Diagram of system feature Find Similar Feed(s)}
\label{fig:find}
\end{figure}
When a user has a feed and wants to find similar feeds he can through the user interface trigger a search. The relationship required to perform this feature is visualized in figure~\ref{fig:find}. The search will look in online feed databases, through their web-based API's, based on data from the feed in form of name and address. \\
\begin{tabular}{l l}
\begin{minipage}{0.5\textwidth}
\begin{enumerate}
  \item Access the feed view, shown in figure~\ref{fig:find_screen}.
  \item Press \textit{Find Similar Feeds}, shown as the first icon in the action bar visualized as a globe.
  \item In the displayed dialog chose which similar feed you like to add.
\end{enumerate}
\end{minipage}
&
\begin{minipage}{0.5\textwidth}
  \centering
  \includegraphics[width=0.8\textwidth]{./images/ViewReadPosts.png}
  \captionof{figure}{Screenshot of the feed view}
  \label{fig:find_screen}
\end{minipage}
\end{tabular}


\newpage
\subsubsection{Manage Feed(s)}
\begin{figure}[hbt]
\centering
\includegraphics[width=0.5\textwidth]
{./images/ManageFeeds.png}
\caption{Class Diagram of system feature Manage Feed(s)}
\label{fig:manage}
\end{figure}
For a user, or administrator to be exact, to manage feeds he has to do so through the user interface in which there are options and functions to perform all required sub-features. The relationship required to perform this feature is visualized in figure~\ref{fig:manage}. In this case the user interface is directly connected to the feed list and feed(s). \\
\begin{tabular}{l l}
\begin{minipage}{0.5\textwidth}
\begin{enumerate}
  \item Access the main view, shown in figure~\ref{fig:add_category_screen}.
  \item Press \textit{Add Category}, shown as the first icon in the action bar visualized as a plus sign.
  \item In the displayed dialog chose a name for the category.
\end{enumerate}
\end{minipage}
&
\begin{minipage}{0.5\textwidth}
  \centering
  \includegraphics[width=0.8\textwidth]{./images/AddCategory.png}
  \captionof{figure}{Screenshot of the add category dialog}
  \label{fig:add_category_screen}
\end{minipage}
\end{tabular} \\
\begin{tabular}{l l}
\begin{minipage}{0.5\textwidth}
\begin{enumerate}
  \item Access the category view, shown in figure~\ref{fig:add_feed_screen}.
  \item Press \textit{Add Feed}, shown as the first icon in the action bar visualized as a plus sign.
  \item In the displayed dialog enter the url address to the feed to be added.
\end{enumerate}
\end{minipage}
&
\begin{minipage}{0.5\textwidth}
  \centering
  \includegraphics[width=0.8\textwidth]{./images/AddFeed.png}
  \captionof{figure}{Screenshot of the add feed dialog}
  \label{fig:add_feed_screen}
\end{minipage}
\end{tabular}


\newpage
\subsection{Directory Organization}
\label{sec:directory_organization}
The structure of the project is the default structure of an \textit{Eclipse} project for \textit{Android} using the \textit{ADT}. The source file are located in \textit{/src/} with a substructure based on the \textit{Java} packages they are located in, something that is mandatory for \textit{Android projects}. The used packages has to be completely unique to latter on be accepted into \textit{Google Play}, the online application market for \textit{Android}. 
\begin{table}[hbt]
\begin{center}
    \begin{tabular}{ | l | l |}
    \hline
    \textbf{Directory} & \textbf{File name}\\ \hline
    /src/bit/app/need4feed/ & MainApplication.java\\ \hline
    /src/bit/app/need4feed/ & MainActivity.java\\ \hline
    /src/bit/app/need4feed/ & CategoryActivity.java\\ \hline
    /src/bit/app/need4feed/ & FeedActivity.java\\ \hline
    /src/bit/app/need4feed/ & PostActivity.java\\ \hline
    /src/bit/app/need4feed/ & RssService.java\\ \hline
    /src/bit/app/need4feed/ & AddCategoryDialog.java\\ \hline
    /src/bit/app/need4feed/ & RemoveCategoryDialog.java\\ \hline
    /src/bit/app/need4feed/ & AddFeedDialog.java\\ \hline
    /src/bit/app/need4feed/ & RemoveFeedDialog.java\\ \hline
    /src/bit/app/need4feed/util/ & DatabaseHandler.java\\ \hline
    /src/bit/app/need4feed/util/ & RssHandler.java\\ \hline
    /src/bit/app/need4feed/util/ & OpmlHandler.java\\ \hline
    /src/bit/app/need4feed/type/ & Category.java\\ \hline
    /src/bit/app/need4feed/type/ & CategoryAdapter.java\\ \hline
    /src/bit/app/need4feed/type/ & Feed.java\\ \hline
    /src/bit/app/need4feed/type/ & FeedAdapter.java\\ \hline
    /src/bit/app/need4feed/type/ & Post.java\\ \hline
    /src/bit/app/need4feed/type/ & PostAdapter.java\\ \hline
    \end{tabular}
    \caption{Directory Organization of Android Source Files}\label{tab:directory_organization_source}
\end{center}
\end{table}
Beyond source files, seen in table~\ref{tab:directory_organization_source}, an \textit{Android} project also contains \textit{XML} files used for layouts for activities, dialogs and menus. \textit{XML} files are also used for strings used graphically throughout the program. All of these \textit{XML} files are located under \textit{/res/layout/}, \textit{/res/menu/} and \textit{/res/strings/}. In addition to \textit{XML} files \textit{/res/} also includes image files used for icons and symbols, these are located in their respective \textit{/res/drawable/} directory. All files under \textit{/res/} are resources that are used dynamically during runtime, this design enables \textit{Android} to change the its look while running based on feature such as size of phone or as simple as the current orientation of the screen.
\begin{table}[hbt]
\begin{center}
    \begin{tabular}{ | l | l |}
    \hline
    \textbf{Directory} & \textbf{File name}\\ \hline
	/res/layout/ & *.xml\\ \hline
	/res/menu/ & *.xml\\ \hline
	/res/strings/ & *.xml\\ \hline
    /res/drawable-hdpi/ & *.png\\ \hline
    /res/drawable-ldpi/ & *.png\\ \hline
    /res/drawable-mdpi/ & *.png\\ \hline
    /res/drawable-xhdpi/ & *.png\\ \hline
    \end{tabular}
    \caption{Directory Organization of Android Resource Files}\label{tab:directory_organization_resource}
\end{center}
\end{table}
The directory organization of the resource files are shown in table~\ref{tab:directory_organization_resource}.


\subsection{Class Description}
\label{sec:class_description}

\subsubsection{MainApplication}
\textit{MainApplication} belongs to the \textit{Backend} layer of the architecture but is however not a \textit{Service}. This class extends the \textit{Android} class \textit{Application} and has the purpose to create a structure with a different life-cycle than of the activities, that can pause or terminate at any point and later on be re-created if returned to. With this feature, of being non-terminated until the application is terminated, \textit{MainApplication} gives the functionality to hold truly global variables and instances accessible to all activities through the \textit{Android} virtual machine.
\begin{description}
  \item[Location:] \textit{/src/bit/app/need4feed/MainApplication.java} \hfill
  \item[Class Components:] \hfill
     \begin{itemize}
        \item databaseHandler: DatabaseHandler
        \item getDatabaseHandler(): DatabaseHandler
     \end{itemize}
\end{description}


\subsubsection{MainActivity}
\textit{MainActivity} is the main user interface and the entry screen of the application, it extends \textit{Activity} and belongs to the \textit{User Interfaces} layer of the architecture. The purpose of the class is to create the first view, holding a list of all categories. It also takes the user to the next view when a category has been clicked. \textit{MainActivity} gives the functionality, beyond displaying categories and managing the transition to \textit{CategoryActivity}, to add and remove categories.
\begin{description}
  \item[Location:] \textit{/src/bit/app/need4feed/MainActivity.java} \hfill
  \item[Class Components:] \hfill
     \begin{itemize}
        \item actionBar: ActionBar
        \item categoryListView: ListView
        \item categoryAdapter: CategoryAdapter
        \item databaseHandler: DatabaseHandler
        \item onFinishAddCategoryDialog(): void
        \item onFinishRemoveCategoryDialog(): void
     \end{itemize}
\end{description}


\subsubsection{CategoryActivity}
\textit{CategoryActivity} is the second activity the user gets in contact with, it extends \textit{Activity} and belongs to the \textit{User Interfaces} layer of the architecture. The purpose of the class is to create a list view of feeds for a certain category, thereby the name \textit{CategoryActivity}. It gives the functionality, except to list all feeds of a category, to take the user to a certain feed if clicked on as well as add and remove feeds to that specific category.
\begin{description}
  \item[Location:] \textit{/src/bit/app/need4feed/CategoryActivity.java} \hfill
  \item[Class Components:] \hfill
     \begin{itemize}
        \item actionBar: ActionBar
        \item feedListView: ListView
        \item feedAdapter: FeedAdapter
        \item databaseHandler: DatabaseHandler
        \item categoryId: long
        \item onFinishAddFeedDialog(): void
        \item onFinishRemoveFeedDialog(): void
     \end{itemize}
\end{description}


\subsubsection{FeedActivity}
\textit{FeedActivity} is the third activity the user gets in contact with, it extends \textit{Activity} and belongs to the \textit{User Interfaces} layer of the architecture. The purpose of the class is to create a list view of posts for a certain feed, thereby the name \textit{FeedActivity}. It gives the functionality, except to list all posts of a feed, to take the user to a certain post if clicked on.
\begin{description}
  \item[Location:] \textit{/src/bit/app/need4feed/FeedActivity.java} \hfill
  \item[Class Components:] \hfill
     \begin{itemize}
        \item actionBar: ActionBar
        \item postListView: ListView
        \item postAdapter: PostAdapter
        \item databaseHandler: DatabaseHandler
        \item feedId: long
     \end{itemize}
\end{description}


\subsubsection{PostActivity}
\textit{PostActivity} is the fourth, and last, activity the user gets in contact with, it extends \textit{Activity} and belongs to the \textit{User Interfaces} layer of the architecture. The purpose of the class is to display a specific post and its content, thereby the name \textit{PostActivity}. It gives the functionality, except to display a post, to take the user to the web page of the post through a click on the post heading.
\begin{description}
  \item[Location:] \textit{/src/bit/app/need4feed/PostActivity.java} \hfill
  \item[Class Components:] \hfill
     \begin{itemize}
        \item actionBar: ActionBar
        \item titleTextView: TextView
        \item feedTextView: TextView
        \item contentWebView: WebView
        \item sourceFeed: Feed
        \item displayedPost: Post
        \item databaseHandler: DatabaseHandler
        \item postId: long
     \end{itemize}
\end{description}

	
\subsubsection{RssService}
\textit{RssService} extends \textit{Intent Service} and belongs to the \textit{Backend} layer of the architecture. The purpose of this class is to serve as a background task, periodically fetching the latest posts. Its functionality is built up out off two parts; database access to both fetch all existing feeds and then store all new posts as well as access the actual \textit{RSS} feeds.
\begin{description}
  \item[Location:] \textit{/src/bit/app/need4feed/RssService.java} \hfill
  \item[Class Components:] \hfill
     \begin{itemize}
        \item databaseHandler: DatabaseHandler
        \item rssHandler: RssHandler
     \end{itemize}
\end{description}


\subsubsection{AddCategoryDialog}
\textit{AddCategoryDialog} is a \textit{Dialog} and belongs to the \textit{User Interface} layer of the architecture. The class extends \textit{DialogFragment} to enable all required \textit{Dialog} functionality. The purpose of this class is to create the user interface that enables the user to input a category name and create a category. Furthermore the functionality includes adding the new category to the database and signal the associated \textit{Activity} through \textit{addCategoryDialogListener} which is connected to \textit{MainActivity}.
\begin{description}
  \item[Location:] \textit{/src/bit/app/need4feed/AddCategoryDialog.java} \hfill
  \item[Class Components:] \hfill
     \begin{itemize}
        \item databaseHandler: DatabaseHandler
        \item addCategoryDialogListener: AddCategoryDialogListener
     \end{itemize}
\end{description}


\subsubsection{RemoveCategoryDialog}
\textit{RemoveCategoryDialog} is a \textit{Dialog} and belongs to the \textit{User Interface} layer of the architecture. The class extends \textit{DialogFragment} to enable all required \textit{Dialog} functionality. The purpose of this class is to create the user interface that enables the user to input which category to remove. Furthermore the functionality includes removing the specific category from the database and signal the associated \textit{Activity} through \textit{removeCategoryDialogListener} which is connected to \textit{MainActivity}.
\begin{description}
  \item[Location:] \textit{/src/bit/app/need4feed/RemoveCategoryDialog.java} \hfill
  \item[Class Components:] \hfill
     \begin{itemize}
        \item databaseHandler: DatabaseHandler
        \item removeCategoryDialogListener: RemoveCategoryDialogListener
     \end{itemize}
\end{description}


\subsubsection{AddFeedDialog}
\textit{AddFeedDialog} is a \textit{Dialog} and belongs to the \textit{User Interface} layer of the architecture. The class extends \textit{DialogFragment} to enable all required \textit{Dialog} functionality. The purpose of this class is to create the user interface that enables the user to input a feed link and create a feed. Furthermore the functionality includes adding the new feed to the database, fetch the latest posts for it, and signal the associated \textit{Activity} through \textit{addFeedDialogListener} which is connected to \textit{CategoryActivity}.
\begin{description}
  \item[Location:] \textit{/src/bit/app/need4feed/AddFeedDialog.java} \hfill
  \item[Class Components:] \hfill
     \begin{itemize}
        \item databaseHandler: DatabaseHandler
        \item addFeedDialogListener: AddFeedDialogListener
     \end{itemize}
\end{description}


\subsubsection{RemoveFeedDialog}
\textit{RemoveFeedDialog} is a \textit{Dialog} and belongs to the \textit{User Interface} layer of the architecture. The class extends \textit{DialogFragment} to enable all required \textit{Dialog} functionality. The purpose of this class is to create the user interface that enables the user to input which feed to remove. Furthermore the functionality includes removing the specific feed from the database and signal the associated \textit{Activity} through \textit{removeFeedDialogListener} which is connected to \textit{CategoryActivity}.
\begin{description}
  \item[Location:] \textit{/src/bit/app/need4feed/RemoveFeedDialog.java} \hfill
  \item[Class Components:] \hfill
     \begin{itemize}
        \item databaseHandler: DatabaseHandler
        \item removeFeedDialogListener: RemoveFeedDialogListener
     \end{itemize}
\end{description}


\subsubsection{DatabaseHandler}
\textit{DatabaseHandler} is part of the \textit{Utilities} layer of the architecture and works as a \textit{Singleton} class as it is only initiated within the \textit{MainApplication}. To enable \textit{SQLite} functionality it extends \textit{SQLiteOpenHelper}. The purpose of this class is to handle all queries to and from the database containing all entities. 
\begin{description}
  \item[Location:] \textit{/src/bit/app/need4feed/util/DatabaseHandler.java} \hfill
  \item[Class Components:] \hfill
     \begin{itemize}
        \item db: SQLiteDatabase
		\item addCategory(Category): void 
		\item addFeed(Feed): void
		\item addPost(Post): void
		\item updateCategory(Category): void
		\item updateFeed(Feed): void
		\item updatePost(Post): void
		\item deleteCategory(long): void
		\item deleteFeed(long): void
		\item deletePostOfFeed(long): void
		\item deletePost(long): void
		\item getCategories(): List\textless Category\textgreater
		\item getCategoryNames(): String[]
		\item getAllFeeds(): List\textless Feed\textgreater
		\item getFeeds(long): List\textless Feed\textgreater
		\item getFeedNames(long): String[]
		\item getFeed(long): Feed
		\item getPosts(long): List\textless Post\textgreater
		\item getPost(long): Post
		\item getPostCount(long) int
     \end{itemize}
\end{description}


\subsubsection{RssHandler}
\textit{RssHandler} is part of the \textit{Utilities} layer of the architecture and extends \textit{DefaultHandler} to be able to process \textit{XML} files. The class imports \textit{SAXParser} to help parse tags. The functionality of \textit{RssHandler} is to access the \textit{Internet} and fetch \textit{RSS} \textit{XML} files and process them, either to fetch information about a feed or its posts. The class then stores the information in the database.
\begin{description}
  \item[Location:] \textit{/src/bit/app/need4feed/util/RssHandler.java} \hfill
  \item[Class Components:] \hfill
     \begin{itemize}
        \item db: DatabaseHandler
        \item currentFeed: Feed
        \item currentPost: Post
        \item postList: List\textless Post\textgreater
		\item verifyFeed(String): String 
		\item getFeed(String): Feed
		\item getAllLatestPosts(): void
		\item getLatestPosts(Feed): void
     \end{itemize}
\end{description}


\subsubsection{OpmlHandler}
\textit{OpmlHandler} is part of the \textit{Utilities} layer of the architecture and extends \textit{DefaultHandler} to be able to process \textit{XML} files. The class imports \textit{SAXParser} to help parse tags. The functionality of \textit{OpmlHandler} is to process feed lists written in \textit{OPML} file format, that expends \textit{XML}. It enables both import and export of feed lists.
\begin{description}
  \item[Location:] \textit{/src/bit/app/need4feed/util/OpmlHandler.java} \hfill
  \item[Class Components:] \hfill
     \begin{itemize}
        \item db: DatabaseHandler
		\item ImportFeeds( File file ): void
		\item ExportFeeds( void ): File
     \end{itemize}
\end{description}


\subsubsection{Category}
\textit{Category}, the entity class, of the \textit{Entities} layer of the architecture. The purpose of this class is to symbolize an entity and manage its variables. Also included is a \textit{compareTo} method to deal with sorting of the entity, for categories the sorting is done alphabetically.
\begin{description}
  \item[Location:] \textit{/src/bit/app/need4feed/type/Category.java} \hfill
  \item[Class Components:] \hfill
     \begin{itemize}
        \item id: long
        \item name: String
		\item getId(): long 
		\item setId(long): void
		\item getName(): String 
		\item setName(String): void
		\item compareTo(Category): int
     \end{itemize}
\end{description}


\subsubsection{CategoryAdapter}
\textit{CategoryAdapter} is utility tight connected to its type, \textit{Category}, and is part of the \textit{Utilities} layer of the architecture. Its purpose is to handle lists of categories and present them in \textit{ListViews}.
\begin{description}
  \item[Location:] \textit{/src/bit/app/need4feed/type/CategoryAdapter.java} \hfill
  \item[Class Components:] \hfill
     \begin{itemize}
        \item categoryList: List\textless Category\textgreater
        \item holder: ViewHolder
		\item setCategoryList(List\textless Category\textgreater): void 
     \end{itemize}
\end{description}


\subsubsection{Feed}
\textit{Feed}, the entity class, of the \textit{Entities} layer of the architecture. The purpose of this class is to symbolize an entity and manage its variables. Also included is a \textit{compareTo} method to deal with sorting of the entity, for feeds the sorting is done alphabetically.
\begin{description}
  \item[Location:] \textit{/src/bit/app/need4feed/type/Feed.java} \hfill
  \item[Class Components:] \hfill
     \begin{itemize}
        \item id: long
        \item categoryId: long
        \item title: String
        \item link: String
        \item feedLink: String
        \item description: String
		\item getId(): long 
		\item setId(long): void
		\item getCategoryId(): long 
		\item setCategoryId(long): void
		\item getTitle(): String 
		\item setTitle(String): void
		\item getLink(): String 
		\item setLink(String): void
		\item getFeedLink(): String 
		\item setFeedLink(String): void
		\item getDescription(): String 
		\item setDescription(String): void
		\item compareTo(Feed): int
     \end{itemize}
\end{description}


\subsubsection{FeedAdapter}
\textit{FeedAdapter} is utility tight connected to its type, \textit{Feed}, and is part of the \textit{Utilities} layer of the architecture. Its purpose is to handle lists of categories and present them in \textit{ListViews}.
\begin{description}
  \item[Location:] \textit{/src/bit/app/need4feed/type/FeedAdapter.java} \hfill
  \item[Class Components:] \hfill
     \begin{itemize}
        \item feedList: List\textless Feed\textgreater
        \item holder: ViewHolder
		\item setFeedList(List\textless Feed\textgreater): void 
     \end{itemize}
\end{description}


\subsubsection{Post}
\textit{Post}, the entity class, of the \textit{Entities} layer of the architecture. The purpose of this class is to symbolize an entity and manage its variables. Also included is a \textit{compareTo} method to deal with sorting of the entity, for posts the sorting is done based on the date which is stored in \textit{pubDate}.
\begin{description}
  \item[Location:] \textit{/src/bit/app/need4feed/type/Post.java} \hfill
  \item[Class Components:] \hfill
     \begin{itemize}
        \item id: long
        \item feedId: long
        \item title: String
        \item link: String
        \item description: String
        \item pubDate: String
        \item thumbnail: String
		\item getId(): long 
		\item setId(long): void
		\item getFeedId(): long 
		\item setFeedId(long): void
		\item getTitle(): String 
		\item setTitle(String): void
		\item getLink(): String 
		\item setLink(String): void
		\item getDescription(): String 
		\item setDescription(String): void
		\item getPubDate(): String 
		\item setPubDate(String): void
		\item getThumbnail(): String 
		\item setThumbnail(String): void
		\item compareTo(Post): int
     \end{itemize}
\end{description}


\subsubsection{PostAdapter}
\textit{PostAdapter} is utility tight connected to its type, \textit{Post}, and is part of the \textit{Utilities} layer of the architecture. Its purpose is to handle lists of categories and present them in \textit{ListViews}.
\begin{description}
  \item[Location:] \textit{/src/bit/app/need4feed/type/PostAdapter.java} \hfill
  \item[Class Components:] \hfill
     \begin{itemize}
        \item postList: List\textless Post\textgreater
        \item holder: ViewHolder
		\item setPostList(List\textless Post\textgreater): void 
     \end{itemize}
\end{description}
