\section{Development Environment}
\label{sec:dev_environ}


\subsection{Platform}


\subsubsection{Target Platform}
Smartphone users mostly choose Android OS compared to other OS such as Windows phone and iPhone OS. Major manufacturing giants like Samsung , LG, Motorola, HTC choose Android OS for their smart phones. Smartphone users also choose Android OS ahead of its rivals as it is the best selling smart phone platform in the world. Here are some of the reasons to choose Android.
First of all is that Android is an open source software stack for mobile devices under Google Inc. The maintenance and further development of Android open source code is led by Google. As Android is an open source no manufacturer can control the innovations of another. The mistake of one manufacturer in Android open source code gets corrected by another and there will be no central point of failure. Furthermore Android applications are easily downloaded from either Google Play, the Android application market, or from a third party. Android applications are easily obtained for various purposes as it is an open source software and there are over 150,000 applications from a large community of developers. Android applications are available in both free as well as paid versions of applications, which is beneficial to both developers as well as the users.


\subsubsection{Development Platform}
The environment for the development of Android applications is provided by Google and it's open source which means that any developer can download it from the Android web page. All these tools are gathered in Android Development Tools(ADT)\cite{android-adt}, in which most common used tool for developers is the Software Development Kit(SDK)\cite{android-sdk}.
Android SDK is a tool for Android applications developers which provides the developer all necessary tools such as the API libraries and developer tools necessary to build, test, and debug apps for Android. But also is needed a programming environment because the SDK must be supported from an Integrated Development Environment(IDE) as Eclipse.
Eclipse is one of the most common IDE’s used by developers. In Eclipse it is possible to program in many different languages such as C++, Java, Python, PHP or Perl. This multi-language versatility is due to the possibility to install different plug-in system. The Android ADT is such a plug-in specific for Android.
The most common language to develop Android applications in is Java, but it is also possible to develop in C or C++ with the Android Native Development Kit(NDK)\cite{android-ndk}. Development in C or C++ does however require knowledge of low level Android systems and is not very common compared to Java. It is also possible, but even more uncommon, to develop applications in  Scala, Python, Lua or Perl.


\subsubsection{Cost}
\begin{table}[h]
\label{tab:dev_costs}
\begin{center}
    \begin{tabular}{ | l | c | c |}
    \hline
    \textbf{Item} & \textbf{Average Estimation} & \textbf{Actual Cost} \\ \hline
    \textit{Development Environment} &  &  \\ 
    Android ADT & Free & 0\$ \\ 
    Eclipse & Eclipse Public License\cite{eclipse-cost} & 0\$ \\
    Google Code & Free & 0\$ \\ \hline
    \textit{Developers} &  &  \\
    Group Members & 25\$/h\cite{dev-cost} & 0\$ \\
    Open-Source Community & Free & 0\$ \\ \hline
    Computers & 500\$/person\cite{comp-cost} & 0\$ \\ \hline
    Internet & 20\$/month\cite{internet-cost} & 0\$ \\ \hline \hline
    \textbf{Total:} & \textbf{10580\$} & \textbf{0\$} \\ \hline
    \end{tabular}
    \caption{Costs of Development}
\end{center}
\end{table}
The cost of development is divided into a set of areas; development environment, developers, computers and Internet. For the actual environment we have the IDE, Eclipse, combined the the Android ADT plug-in which are both free to use for open-source projects such as this so there is no cost. The project is hosted, including revision handling and issue tracker, on Google Code which is also free for open-source projects bringing the total cost of the development environment to nothing. Developers in this case can be divided into the group members and any volunteer from the open-source community. If the group members had been paid developers an average cost estimation of 25\$/h could be used for each developer. Combine that with 40/5 hours per week, standard work week divided by amount of courses, and 15 weeks and it would result in 3000\$/developer of the project time frame. However, as the project group are students and are doing this as a school project there is no cost. For the equipment, computers and Internet, standard rates has been used for the average estimation. However, similar to the developer cost the students already have computers and Internet access as they are located now, even if they didn't have it themselves they would have had access at school. The total estimated cost would then result in 10580\$ while the actual cost for this project is 0\$.


\subsection{Task Distribution}
\begin{description}
  \item[Brian Pohl] \hfill \\
  	\begin{itemize}
  	\item Demonstration
  	\item Modelling
  	\item Design
	\end{itemize}
  \item[Iker Trun] \hfill \\
    \begin{itemize}
  	\item Programming: Database
  	\item Development Environment
  	\item Testing
	\end{itemize}
  \item[Thomas Ingvarsson] \hfill \\
    \begin{itemize}
  	\item Programming
  	\item Programming: User Interfaces
  	\item Programming: Feed Access
  	\item Documentation
  	\item Version Handling
  	\item Issue Tracker
	\end{itemize}
\end{description}